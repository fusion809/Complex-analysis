\documentclass[12pt,a4paper]{article}
\usepackage{gensymb}
\usepackage{amsmath}
\usepackage{amssymb}
\usepackage{enumitem}
\usepackage{graphicx}
\usepackage{setspace}
\usepackage{float}
\usepackage{listings}
\usepackage{multirow}
\usepackage{mathtools}
\usepackage[margin=0.5in]{geometry}
\newcommand{\inlinecode}{\texttt}
\newcommand{\question}{\textbf{Question}: }
\newcommand{\itq}{\item\question }
\newcommand{\answer}{\textbf{Answer}: }
\newcommand{\alphlist}{\begin{enumerate}[label=(\alph*)]}
\setlength{\parindent}{0pt}
\linespread{1.5}
\newcommand{\labsec}[1]{\section{#1}
	\label{sec:#1}}
\newcommand{\seclin}[1]{\hyperref[sec:#1]{\caselower{#1}}}
\newcommand{\defeq}{\vcentcolon=}
\newcommand{\eqdef}{=\vcentcolon}

\begin{document}
	{\Huge \textbf{Complex analysis notes}}
	
	\tableofcontents
	\newpage
	
	{\LARGE \textbf{Glossary}}
	
	\labsec{Analytic}
	
	A function is analytic at a point, if at that point and within its neighbourhood, its derivative is defined.
	
	\labsec{Branch point}
	
	A branch point is a singularity across which the function is discontinuous. $\ln{z}$ is an example of a discontinuous function, as at $z=0$ it jumps from having $\pi i$ in it and not having that term.
	
	\labsec{Cauchy-Riemann conditions}
	
	The Cauchy-Riemann conditions determine whether a function is differentiable. Assuming a complex function, $f(z)$, can be written in the form:
	\begin{align*}
		f(z) = u(x,y) + iv(x,y)
	\end{align*}

	Then the Cauchy-Riemann conditions are:
	\begin{align*}
		\dfrac{\partial u}{\partial x} &= \dfrac{\partial v}{\partial y} \\
		\dfrac{\partial u}{\partial y} &= - \dfrac{\partial v}{\partial x}
	\end{align*}

	It follows that every $u$ and $v$ that satisfies these conditions also satisfy Laplace's equation, that is:
	\begin{align*}
	\Delta u &\defeq \nabla \cdot \nabla u = 0 \\
	\Delta v &= 0
	\end{align*}

	\labsec{Continuous function}
	
	A function, $f(z)$, is continuous at a point $z_0$, if it satisfies the following criteria:
	\[
		f(z_0) \hspace{0.1cm}\mathrm{exists,}
	\]
	\[
		\lim_{z\rightarrow z_0} f(z) \hspace{0.1cm}\mathrm{exists,}
	\]
	\[
		\lim_{z\rightarrow z_0} f(z) = f(z_0)
		\hspace{0.1cm}\mathrm{exists}
	\]
	
	If $f = u + iv$ is continuous, then both $u$ and $v$ must be continuous.

	\labsec{Derivative of a complex function}
	
	The derivative of a complex function, $f(z)$, is:
	\begin{align*}
		f'(z) &= \dfrac{df}{dz} \\
		&= \dfrac{\partial u}{\partial x} + i \dfrac{\partial v}{\partial x}
	\end{align*}

	In polar form, with $z = re^{i\theta}$, $f'(z)$ is defined as:
	\[
	f'(z) = e^{-i\theta} \left(\dfrac{\partial u}{\partial r}+i \dfrac{\partial v}{\partial r}\right)
	\]
	
	\labsec{Entire function}
	
	A holomorphic function whose domain is the whole complex plane is called an entire function.
	
	\labsec{Essential singularity}
	
	An isolated singularity that is not a pole nor a removable singularity is called an \textbf{essential singularity}.
	
	\labsec{Holomorphic function}
	
	A holomorphic function is a function that is analytic within a neighbourhood of every point within its domain.
	
	\labsec{Isolated singularity}
	
	A singularity, $z_0$, is isolated if and only if the circle $|z-z_0|=\delta$ encloses no singularities other than $z_0$ itself. 

	\labsec{Neighbourhood}
	
	The neighbourhood of a point is essentially a disk of size $\epsilon$ around said point. In other words, it is defined by all $z$ values such that:
	\[
	|z-z_0| < \epsilon
	\] 
	
	\labsec{Pole}
	
	If $z_0$ is an isolated singularity, and there exists a positive integer $n$ such that:
	\[
	\lim_{z\rightarrow z_0} (z-z_0)^n f(z) \neq 0
	\]
	then $z_0$ is said to be a \textbf{pole of order} $\mathbf{n}$. 
	
	\labsec{Removable singularity}
	
	An isolated singularity, $z_0$, is removable if and only if:
	\[
	\lim_{z \rightarrow z_0} f(z)
	\]
	exists.
	
	\labsec{Residue}
	
	Every pole (which will be denoted by $a_j$) has an associated residue (which we will call $r_j$). It is defined by:
	
	\begin{align}
		r_j = \lim_{z\rightarrow a_j} \dfrac{1}{(k_j-1)!} \dfrac{d^{k_j-1}}{dz^{k_j-1}} \left[(z-a_j)^{k_j} f(z)\right]
	\end{align}

	where $k_j$ is the order of the $j$th pole.

	\labsec{Residue theorem}
	
	The residue theorem states that:
	\begin{align*}
		\oint_C f(z) dz &= 2\pi i \sum_j r_j \\
		&= 2\pi i \sum_j \lim_{z\rightarrow a_j} \dfrac{1}{(k_j-1)!} \dfrac{d^{k_j-1}}{dz^{k_j-1}} \left[(z-a_j)^{k_j} f(z)\right]
	\end{align*}
\end{document}